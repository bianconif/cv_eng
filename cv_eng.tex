\documentclass[11pt]{article} 

%\usepackage{helvet} % Default font is the helvetica postscript font
%\usepackage{newcent} % To change the default font to the new century schoolbook postscript font uncomment this line and comment the one above

%----------------------------------------------------------------------------------------
%	CUSTOM PACKAGES
%----------------------------------------------------------------------------------------
\usepackage{amsmath}
\usepackage{calc}
\usepackage{hyperref}
\usepackage{enumitem}
\usepackage{fontawesome}
\usepackage[a4paper, total={6.5in, 9in}]{geometry}
\usepackage{marvosym}
\usepackage{tabularx}

%For the references
\usepackage{natbib}
\usepackage{bibentry}

%----------------------------------------------------------------------------------------
%----------------------------------------------------------------------------------------
%----------------------------------------------------------------------------------------

%----------------------------------------------------------------------------------------
%	CUSTOM COMMANDS
%----------------------------------------------------------------------------------------
%Set Sans Serif fonts
\renewcommand{\familydefault}{\sfdefault}
%----------------------------------------------------------------------------------------
%----------------------------------------------------------------------------------------
%----------------------------------------------------------------------------------------

%Pointer to bib file
\nobibliography{publications}

\begin{document}

%----------------------------------------------------------------------------------------
%	Header
%----------------------------------------------------------------------------------------
\begin{center}
\large{\bf Francesco BIANCONI} \\
\emph{Curriculum vitae}
\end{center}
\vspace{\baselineskip}

\begin{center}
\begin{small}
%\begin{tabular}{l@{\hskip 0.5in}l|l}
\begin{tabular}{ll@{\hskip 1.5cm}l}
\bf Office & \bf Home \\
Department of Engineering & Via Luigi Catanelli 54 & \Email \ \href{mailto:bianco@ieee.org}{\texttt{bianco@ieee.org}}\\
Università degli Studi di Perugia & 06135 Perugia, Italy & \faGlobe \ \url{www.bianconif.net}\\
Via Goffredo Duranti 93 & \Telefon \ +39 075 966 1582 & \faGithub \ \href{https://github.com/bianconif}{\texttt{bianconif}} \\
06125 Perugia, Italy & \Mobilefone \ +39 347 585 9738 \\
\Telefon \ +39 075 585 3706\\
\end{tabular}
\end{small}
\end{center}
%----------------------------------------------------------------------------------------
%	End of header
%----------------------------------------------------------------------------------------

\rule{\textwidth}{0.75pt}

\section*{Personal details}

Born: 17 July 1971, Perugia, Italy\\
Gender: Male \\
Citizenship: Italian 

\section*{Education} 

\subsection*{University degrees}

\begin{itemize}
\item Doctor of Philosophy, 
\emph{Computer-aided Mechanical Design} \\ 
Università degli Studi di Perugia, Perugia, Italy \hfill Jan. 2001 

\item Master of Engineering, \emph{Mechanical Engineering} \\ 
Università degli Studi di Perugia, Perugia, Italy \hfill Apr. 1997
\end{itemize}

\subsection*{Other courses}

\begin{itemize}

\item 
\emph{Applied Data Science Specialization}, IBM/Coursera \hfill Jun. 2020

\end{itemize}

\section*{Languages} 

\begin{itemize}

	\item English: Advanced
	\begin{itemize}
		\item Cambridge CB CAE (grade A -- 81/100) \hfill Apr. 2013
		\item iBT TOEFL (104/120) \hfill Dec. 2011
	\end{itemize}
	\item Spanish: Advanced
	\begin{itemize}
		\item DELE intermediate (95/100) \hfill May 2006
	\end{itemize}\textbf{}
	\item Italian: Native
	
\end{itemize}

\section*{IT \& Programming skills} 

\begin{itemize}
	\item Programming languages \& VCS
	\begin{itemize}
		\item C, C++, Java, Python and Git
	\end{itemize}
\end{itemize}

\begin{itemize}
	\item Digital typesetting \& office automation
	\begin{itemize}
		\item Microsoft Excel, Microsoft Word and \LaTeX
	\end{itemize}
\end{itemize}

\begin{itemize}
	\item Scientific packages \& data visualisation tools
	\begin{itemize}
		\item Matlab, Mathematica and Tableau
	\end{itemize}
\end{itemize}

\begin{itemize}
	\item CAD/CAE
	\begin{itemize}
		\item Autodesk AutoCAD, Autodesk Inventor and SolidWorks
	\end{itemize}
\end{itemize}


\section*{Employment}

\begin{itemize}
	\item \emph{Associate professor}, Department of Engineering, Università degli Studi di Perugia, Italy \\ \mbox{} \hfill Jul. 2015--present
	\item \emph{Assistant professor}, Department of Engineering, Università degli Studi di Perugia, Italy \\ \mbox{} \hfill Nov. 2000--Jun. 2015
	\item \emph{CAD Engineer} (internship), Umbra Cuscinetti SpA, Foligno (Italy) \mbox{} \hfill Oct.--Nov. 1998
	\item \emph{Secondary school teacher} (Textile Technologies), Istituto Professionale Statale per l’Industria e L’Artigianato, Perugia (Italy) \mbox{} \hfill Dec. 1997--Jun. 1998
	\item \emph{Secondary school teacher} (Civil and Industrial Plants), Istituto Tecnico Statale per Geometri, Perugia (Italy) \mbox{} \hfill Dec. 1997--Jun. 1998
	\item \emph{Secondary school teacher} (Applied Mechanics), Istituto Tecnico Industriale Statale, Foligno (Italy) \mbox{} \hfill Jun.--Jul. 1997
\end{itemize} 

\section*{Visiting positions}

\begin{itemize}
	\item \emph{Academic visitor}, School of Mathematics, Computer Science and Engineering; City, University of London, United Kingdom \mbox{} \hfill Sep.--Dec. 2018
	\item \emph{Academic visitor}, School Electronic Engineering and Computer Science; Queen Mary, University of London, United Kingndom \mbox{} \hfill Sep.--Dec. 2015
	\item \emph{Visiting research fellow}, School of Computing Sciences; University of East Anglia, United Kingdom \mbox{} \hfill Oct.--Dec. 2010
	\item \emph{Visiting researcher}, School of Industrial Engineering, Department of Engineering Design; Universidade de Vigo, Spain \mbox{} \hfill Jun.--Jul. 2009
	\\ \mbox{} \hfill Sep. 2007
	\\ \mbox{} \hfill Sep. 2006
	\\ \mbox{} \hfill Sep. 2005
\end{itemize}

\section*{Teaching}

\subsection*{Undergraduate courses}

\begin{itemize}
	\item \emph{Computer skills} (2 CFU\footnote{Stands for \emph{Credito Formativo Universitario}. This is the unit used in the Italian higher education system to measure and assess the student's work and effort. The minimum number of required credits is 180 for a Bachelor's and 120 for a Master's degree.}), Università degli Studi di Perugia, Department of Engineering, BEng Mechanical Engineering \mbox{} \hfill 2012--present
	
	\item \emph{Technical Drawing} (5 CFU) + \emph{Computer skills} (2 CFU), Università degli Studi di Perugia, Department of Engineering, BEng Industrial Engineering \mbox{} \hfill 2005--present

	\item \emph{Technical Drawing} (6 CFU) + \emph{Computer skills} (2 CFU), Università degli Studi di Perugia, Department of Engineering, BEng Management Engineering \mbox{} \hfill 2017--present
	
	\item \emph{Machine Drawing} (10 CFU) + \emph{Computer skills} (2 CFU), Università degli Studi di Perugia, Department of Engineering, BEng Mechanical Engineering \mbox{} \hfill 2008--2012
  
	\item \emph{Machine Drawing} (6 CFU) Università degli Studi di Perugia, Department of Engineering, BEng Mechanical Engineering \mbox{} \hfill 2000--2007
  \item \emph{Technical Drawing} (5 CFU) Università degli Studi di Perugia, Department of Engineering, BEng Materials Engineering \mbox{} \hfill 2000--2004
 
 \item \emph{CAD Laboratory} (2 CFU) Università degli Studi di Perugia, Department of Engineering, BEng Materials Engineering \mbox{} \hfill 2003–2005
\end{itemize}

\subsection*{Post-graduate courses}

\begin{itemize}
	\item \emph{Product Design and Development} (6 CFU), Università degli Studi di Perugia, Department of Engineering, MEng Industrial Engineering \mbox{} \hfill 2009--2010
	\item \emph{Design Methods of Industrial Engineering} (6 CFU), Università degli Studi di Perugia, Department of Engineering, MEng Mechanical Engineering \mbox{} \hfill 2008--2009
\end{itemize}

\subsection*{Post-master courses and continuous education}

\begin{itemize}
	\item \emph{Computer-aided Design} (2,75 CFU), Università degli Studi di Perugia, Graduate Teacher Training Programme \mbox{} \hfill 2005
	\item \emph{Computer-aided Design} (4 CFU), Università degli Studi di Perugia, Faculty of Engineering post-master course in Materials Engineering \mbox{} \hfill 2004
  \item \emph{Technical Drawing I} (1,75 CFU), Università degli Studi di Perugia, Graduate Teacher Training Programme \mbox{} \hfill 2002
  \item \emph{Technical Drawing II} (2 CFU), Università degli Studi di Perugia, Graduate Teacher Training Programme \mbox{} \hfill 2003
\end{itemize}

\subsection*{Short courses}

\begin{itemize}
	\item \emph{Fundamentals of Engineering Drawing: Theory and Applications} (10h), FAIST Componenti S.p.A, Montone, Italy, \mbox{} \hfill Feb. 2011
	\item \emph{Fundamentals of Pattern Recognition and Image Processing} (10h), Universidade de Vigo, Spain, Doctoral Programme in Environmental Engineering \mbox{} \hfill Jun. 2009
	\item \emph{Introduction to Technical Drawing} (15h), Black \& Decker Italia, Corciano, Italy \\ \mbox{}  \hfill Dec. 2008--Jan. 2009
	\item \emph{Introduction to CAD/CAE} (10h), Master in Virtual Engineering, ITT s.c.a.r.l, Umbertide, Italy \\ \mbox{} \hfill 2005
	\item \emph{Introduction to Object-oriented Programming in C/C++}, Università degli Studi di Perugia, Italy, Doctoral Programme in Industrial Engineering \mbox{} \hfill Apr.--May 2004
\end{itemize}

\subsection*{Teaching in Erasmus interchange programmes}

\begin{itemize}
	\item \emph{Expresión Gráfica} (Technical Drawing, 6h). BSc Energy Engineering and Mining, and BSc Energy Resources and Engineering, Universidade de Vigo, Spain \hfill Dec. 2013 
	\item \emph{Expresión Gráfica} (Technical Drawing, 6h). BSc Engineering of Industrial Technologies, BSc Management Engineering, BSc Electrical Engineering, BSc Mechanical Engineering, BSc Industrial Automation and Electronic Engineering, and BSc Industrial Chemical Engineering, Universidade de Vigo, Spain \hfill Jan. 2012
	\item \emph{Expresión Gráfica} (Technical Drawing, 8h). BSc in Industrial Engineering, Universidade de Vigo, Spain \hfill Apr. 2019
\end{itemize}

\subsection*{Talks}

\begin{itemize}
	\item \emph{Advances in modelling and analysis of the human body by computational imaging} (with G. Pascoletti). Keynote lecture, 2$^\text{nd}$ International Congress on Engineering Sciences and Multidisciplinary Approaches, Istanbul, Turkey \hfill 18~Sep.~2021
	\item \emph{Radiomics in medical imaging: an overview}. Invited talk, IET Webinar Recent advances in Medical Image Analysis \hfill 25 Jun. 2021
	\item \emph{Texture and colour descriptors for visual recognition: historical overview and applications to computer vision and robotics}. Keynote lecture, The 2020 International Conference on Control, Automation and Diagnosis (ICCAD’20), Paris, France \hfill 7 Oct. 2020
	\item \emph{Role of artificial intelligence techniques (automatic classifiers) in molecular imaging modalities in neurodegenerative diseases} (with B. Palumbo). Invited talk, short course in Big Data, Radiomics \& Artificial Intelligence; Italian Association for Medical Physics (AIFM), Reggio Emilia, Italy \hfill 15-16 Dec. 2017
	\item \emph{Towards a procedural model for CAD data exchange}; 5$^\text{th}$ workshop on Design Tools and Methods in Industrial Engineering, Pisa, Italy, \hfill 21-23 Mar. 2005
	\item \emph{Collaborative CAD modeling and construction of augmented CAD models}; 4$^\text{th}$ workshop on Design Tools and Methods in Industrial Engineering, Erice, Italy \hfill 29 Sep.--1 Oct. 2003
	\item \emph{Approaches for integration of CAD/CAM/CAE systems}; 3$^\text{rd}$ Workshop on Design Tools and Methods in Industrial Engineering, Firenze, Italy \hfill 27-28 Jun. 2002
	\item \emph{Interface-based methods for data exchange among CAx systems}; 2$^\text{nd}$ Workshop on Design Tools and Methods in Industrial Engineering, Perugia, Italy \hfill 5-6 Jul. 2001
	\item \emph{Collaborative design and data exchange through STL files}; 1$^\text{st}$ Workshop on Design Tools and Methods in Industrial Engineering, Parma, Italy \hfill Sep. 2000
\end{itemize}

\subsection*{Seminars}

\begin{itemize}
	\item \emph{Texture and colour descriptors for visual recognition: an overview of methods applications}. Doctoral programme in Industrial and Information Engineering, Università degli Studi di Perugia, Italy \hfill 23~Jun.~2021
	\item \emph{Hand-designed descriptors vs. pre-trained convolutional networks: a comparison of two strategies for colour texture classification}. School of Mathematics, Computer Science and Engineering; City, University of London, United Kingdom \hfill 20~Nov.~2018
	\item \emph{Texture description through histograms of equivalent patterns: A unifying Framework for LBP and related methods}. School of Computing Sciences, University of East Anglia, United Kingdom \hfill 9~Dec.~2015
	\item \emph{Texture description through histograms of equivalent patterns: A unifying Framework for LBP and related methods}. School of Electronic Engineering and Computer Science; Queen Mary, University of London, United Kingdom \hfill 22 Sep. 2015
	\item \emph{Introduction to computer vision}, School of Industrial Engineering, Universidade de Vigo, Spain \\ \mbox{} \hfill 21 Jan. 2014
	\item \emph{Fundamentals of pattern recognition and colour image analysis}, School of Industrial Engineering, Universidade de Vigo, Spain \hfill 27~Nov.~2012
	\item \emph{Introduction to pattern recognition and computer vision with applications in the industry}, School of Industrial Engineering, Universidade de Vigo, Spain \hfill 24 Jan. 2012
	\item \emph{Colour vision and pattern recognition}. School of Industrial Engineering, Universidad de Vigo, Spain \hfill 11 Jan. 2011
	\item \emph{Automatic characterization of materials appearance through texture and colour analysis}. School of Computing Sciences, University of East Anglia, United Kingdom \hfill 15 Oct. 2010
	\item \emph{Data exchange among CAD/CAM/CAE systems: problems and perspectives}. Università degli Studi dell’Aquila, Italy \hfill 16 Dec. 2004
\end{itemize} 

\subsection*{Tutorials}

\begin{itemize}
	\item \emph{Colour texture analysis and classification} (with C. Cusano and P. Napoletano), 5$^\text{th}$ Computational Colour Imaging Workshop (CCIW’17), Milan, Italy \hfill 29 Mar. 2017
\end{itemize}

\subsection*{Supervision of PhD dissertations}

\begin{itemize}
	\item R. Bello-Cerezo \emph{Colour texture classification at the end of the `early' years: hand-designed descriptors or pre-trained convolutional neural networks?} Doctoral Programme in Industrial and Information Engineering, Università degli Studi di Perugia, Italy \hfill Apr. 2019
\end{itemize}

\subsection*{Supervision of BSc and MSc theses}

\begin{itemize}
	\item Thirty BSc and MSc theses within the Department of Engineering, Università degli Studi di Perugia, Italy
	\item Eight MSc theses within the School of Industrial Engineering, Universidade de Vigo, Spain
\end{itemize}

\section*{Research projects}

\subsection*{As principal investigator}

\begin{itemize}
	\item \emph{Caratteristiche di forma, colore e tessitura per l’analisi di immagini piane e volumetriche: metodi ed applicazioni} (\emph{Shape, colour and texture features for the analysis of two- and three-dimensional images: methods and applications}). Fundamental research grants, Department of Engineering, Università degli Studi di Perugia, Italy. Amount granted: € 2.751,32. \hfill 2020-2021
\end{itemize}

\subsection*{As investigator/participant}

\begin{itemize}
	\item \emph{Artificial intelligence for Earth observation}. Fundamental research grants, Department of Engineering, Università degli Studi di Perugia, Italy. Amount granted: € 3000,00. \hfill 2021-2022
	
	\item \emph{Algoritmi  classici e  di Machine  Learning per lo sviluppo di modelli sperimentali ``data driven'' per applicazioni robotiche e per la classificazione di prodotti industriali} (\emph{Traditional and Machine Learning algorithms for developing data-driven robotics applications and for automatic classification of industrial products}). Fundamental research grants, Department of Engineering, Università degli Studi di Perugia, Italy. Amount granted: € 5600,00. \hfill 2019-2020
	
	\item \emph{Identificación basada en objetos de cultivos hortícolas bajo invernadero a partir de stereo imágenes del satélite WorldView-3 y series temporales de Landsat-8 -- Ref. AGL2014-56017-R} (\emph{Object-based identification of greenhouse horticultural crops through satellite stereo imagery from WorldView-3 and time series from Landsat-8}). Ministry of Economy and Competitiveness, Spain; Universidad de Almería, Spain. Amount granted: € 85.000,00. \hfill 2015-2018
	
	\item \emph{BioMeTron: Un nuovo approccio integrato e multidisciplinare per lo studio, la gestione e la progettazioni di impianti biologici per l'energia e l'ambiente} (\emph{A new multi-disciplinary, integrated approach for studying, designing and managing biological plants for energy and the environment}). Fundamental research grants, Department of Engineering, Università degli Studi di Perugia, Italy. Amount granted: € 14.000,00. \hfill 2015-2017

	\item \emph{LIFE12 ENV/IT/000411 Enhanced material recovery and environmental sustainability for small scale waste management systems}. European Commission. \hfill 2013-2018
	
	\item \emph{GEOEYE1-WV2: Generación de datos georeferenciados de muy alta resolución a partir de imágenes de los satélites GeoEye-1 Y WorldView-2 -- ref. CTM2010-16573} (\emph{Generation of high resolution geo-referenced data from GeoEye-1 and WorldView-2 satellite images}). Ministry of Science and Education, Spain; Universidad de Almería, Spain. \hfill 2011-2013
	
	\item \emph{EFESO: Environmental Friendly Energy from Solid Oxide Fuel Cells}. Ministry of Economic Development, Italy. Total amount granted: € 10.922.360,00 \hfill 2009-2011
	
	\item \emph{Modeling, simulation and experimental evaluation of materials for Industrial and Civil Engineering}. Consortium for the development of the University District, Terni, Italy. Amount granted: € 73.000,00 \hfill 2007-2008
	
	\item \emph{Sistemi innovativi per la gestione e la condivisione delle informazioni di prodotto in ambienti CAx (Innovative systems for sharing and managing product data in CAx environments)}. Ministry of Education, University and Research; PRIN programme 2005, Italy. Amount granted: € 73.000,00 \hfill 2007-2008

\end{itemize}

\section*{Other grants and fellowships for teaching, research and consultancy}

\begin{itemize}
	\item \emph{FFABR: Finanziamento delle Attività Base di Ricerca} (\emph{Funding for Basic Activities Related to Research}). National Agency 
for the Evaluation of Universities and Research Institutes, Italy. Amount: € 3000,00. \hfill 2017
\end{itemize}

\begin{itemize}
	\item \emph{Introduction to Engineering Drawing: Theory and applications}. Faist Componenti S.p.A., Montone (PG), Italy. Grant type: teaching and consultancy. Amount: € 3000,00. \hfill 2018
\end{itemize}

\begin{itemize}
	\item \emph{Introduction to Engineering Drawing: Theory and applications}. Faist Componenti S.p.A., Montone (PG), Italy. Grant type: teaching and consultancy. Amount: € 1500,00. \hfill 2017
\end{itemize}

\begin{itemize}
	\item \emph{Introduction to Engineering Drawing: Theory and applications}. Faist Componenti S.p.A., Montone (PG), Italy. Grant type: teaching and consultancy. Amount: € 1000,00. \hfill 2016
\end{itemize}

\begin{itemize}
	\item \emph{Introduction to Computer Vision}. Universidade de Vigo, Spain. Grant type: teaching. Amount: € 500,00. \hfill 2014
\end{itemize}

\begin{itemize}
	\item \emph{Fundamentals of pattern recognition and colour image analysis}. Universidade de Vigo, Spain. Grant type: teaching. Amount: € 500,00. \hfill 2013
\end{itemize}

\begin{itemize}
	\item \emph{Fundamentals of pattern recognition and colour image analysis}. Universidade de Vigo, Spain. Grant type: teaching. Amount: € 1200,00. \hfill 2012
\end{itemize}

\begin{itemize}
	\item \emph{Colour vision and pattern recognition}. Universidade de Vigo, Spain. Grant type: teaching. Amount: € 1200,00. \hfill 2011
\end{itemize}

\begin{itemize}
	\item \emph{Digital image processing and pattern recognition}. Universidade de Vigo, Spain. Grant type: teaching. Amount: € 1800,00. \hfill 2009
\end{itemize}

\begin{itemize}
	\item \emph{Introduction to Engineering drawing, materials and manufacturing processes}. Black \& Decker Italia S.p.A., Corciano (PG), Italy. Grant type: teaching and consultancy. Amount: € 6000,00. \\ \mbox{} \hfill 2008-2009
\end{itemize}

\begin{itemize}
	\item \emph{Characterization of the visual appearance of ornamental stone through combination of classifiers}. Universidade de Vigo, Spain. Grant type: visiting fellowship. Amount: € 1200,00. \hfill 2006
\end{itemize}

\begin{itemize}
	\item \emph{Characterization of the visual appearance of natural stones (marble and granite) through image processing and artificial intelligence}. National Research Council, Italy. Grant type: Short-term mobility fellowship. Amount: € 1.787,87. \hfill 2005
\end{itemize}

\begin{itemize}
	\item \emph{Design and optimisation of shelters}. O.M.C. Srl, Passignano (PG), Italy. Grant type: consultancy. Amount: € 5.760,00. \hfill 2005
\end{itemize}

\section*{Editorial work}

\begin{itemize}
	\item Academic Editor for \emph{PLOS One} (e-ISSN: 1932-6203) \hfill Jun. 2020-present
	\item Academic Editor for \emph{Applied Sciences} (e-ISSN: 2076-3417) \hfill Nov. 2020-present
	\item Guest Editor of the Special Issue \emph{Artificial intelligence in Image-Based Diagnostics of Oncological and Neurological Disorders}, Diagnostics (ISSN 2075-4418) \hfill 2020-2021
\end{itemize}

\section*{Publications} 

\subsection*{Selected (sorted by date, newest first)}

\begin{itemize}[label={}]
	\item \bibentry{Bianconi2021d}
\end{itemize}

\subsection*{Edited books}

\begin{enumerate}[label={[\arabic*]}]
	\item \bibentry{ColourTextureBook-2021}
\end{enumerate}

\subsection*{Book chapters}

\begin{enumerate}[label={[\arabic*]}]
	\item \bibentry{Springer-BookChapter-2014}
	\item \bibentry{Springer-ISRL-2020}
\end{enumerate}

\subsection*{Journal papers}

\begin{enumerate}[label={[\arabic*]}]
	\item \bibentry{Bianconi2021e}
	\item \bibentry{Chirikhina2021}
	\item \bibentry{Palumbo2021}
	\item \bibentry{Bianconi2021b}
	\item \bibentry{Bianconi2021}
	\item \bibentry{Bianconi2021c}
	\item \bibentry{Palumbo2021a}
	\item \bibentry{Bianconi2021d}
	\item \bibentry{Bianconi2020b}
	\item \bibentry{Palumbo2020}
	\item \bibentry{Chirikhina2020}
	\item \bibentry{Buratti2020}
	\item \bibentry{Palumbo2020a}
	\item \bibentry{Nuvoli2020}
	\item \bibentry{Bianconi2020}
	\item \bibentry{Smeraldi2020}
\end{enumerate}

\subsection*{Conference proceedings}

\begin{enumerate}[label={[\arabic*]}]
	\item \bibentry{Bianconi2020a}
	\item \bibentry{Bianconi:2019c}
\end{enumerate}





\bibliographystyle{unsrt} 
\nobibliography{publications}




\end{document}